\chapter{TÍTULO DO CAPÍTULO...}{}
\label{cap:03}

\lipsum

\definecolor{verde}{RGB}{25, 90, 25}
\lstset{
	language=c,
	extendedchars=true, % permite acentos
	basicstyle=\ttfamily\scriptsize, 
	keywordstyle=\color{blue}, 
	stringstyle=\color{purple}, 
	commentstyle=\color{verde}, 
	extendedchars=true, 
	showspaces=false, 
	showstringspaces=false, 
	numbers=left,
	numberstyle=\ttfamily\tiny,
	breaklines=true, 
	backgroundcolor=\color{white},
	breakautoindent=true, 
	captionpos=t, %top caption
	xleftmargin=10.5mm,
	tabsize=3,
	framexleftmargin=9.5mm,
	framexrightmargin=-1.0mm, 
	frame=shadowbox, % 〈none|leftline|topline|bottomline|lines|single|shadowbox〉
	rulesepcolor=\color{gray},
	numberbychapter=false,
	backgroundcolor=\color{white},
	firstnumber=86,
	%frameround=tttt
}
\subsection{Ambiente}

O ambiente \textbf{lstlisting} pode ser utilizado para inserção de trecho de códigos de programas importantes para a compreensão do seu trabalho. Ele foi configurado da classe \corujatex para ser designado como \textbf{Programa}. Um exemplo pode observado no \autoref{prog:01}. Analise também no código \LaTeX o comando 

\begin{lstlisting}[caption= Código teste em Linguagem C, label=prog:01] 
void main() 
{
	char slot;
	//configuracoes basicas - sem acentos
	CMCON=0x07;
	ADCON1=0x0F;
	INTCON.GIE=0;
	TRISB.F0=0;
	PORTB.F0=0;
	InicializaMatriz();
	InicializaTeclado();
	InicializaDisplay();
	while(1)
	{
		PORTB.F0=1;
		Delay_ms(250);
		PORTB.F0=0;
		Delay_ms(250);
	}
}	
\end{lstlisting}

\section{Seção do capítulo}
\lipsum %% gera texto apenas para teste

\begin{lstlisting}[caption= Código teste do MikroC for PIC] 
	void main() 
	{
		char slot;
		//configuracoes basicas - sem acentos
		CMCON=0x07;
		ADCON1=0x0F;
		INTCON.GIE=0;
		TRISB.F0=0;
		PORTB.F0=0;
		InicializaMatriz();
		InicializaTeclado();
		InicializaDisplay();
		while(1)
		{
			PORTB.F0=1;
			Delay_ms(250);
			PORTB.F0=0;
			Delay_ms(250);
		}
	}	
\end{lstlisting}
