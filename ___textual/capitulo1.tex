%\newpage
%\pagestyle{empty}

\chapter{INTRODUÇÃO}{}
%\thispagestyle{empty}
%\pagestyle{fancy}
\section{Teste}

Veja no \autoref{qd:Teste} bla bla bla ...

\begin{quadro}[h]
	\centering
	\caption{Estou colocando um quadro no meu texto.}
	%\vspace{5mm}
	\begin{tabular}{|c|c|}
		\hline
		a & b \\
		\hline
		1 & 2 \\
		\hline
	\end{tabular}
	\label{qd:Teste}
\end{quadro}

Observe na \autoref{tab:excel} bla bla bla ...
% Table generated by Excel2LaTeX from sheet 'Planilha1'
\begin{table}[htbp]
	\centering
	\caption{Tabela de teste gerada pela macro Excel2LaTeX}
	\begin{tabular}{rrrr}
		\toprule
		\multicolumn{1}{l}{\textbf{A}} & \multicolumn{1}{l}{\textbf{B}} & \multicolumn{1}{l}{\textbf{C}} & \multicolumn{1}{l}{\textbf{D}} \\
		\midrule
		\rowcolor[rgb]{ .851,  .851,  .851} 1     & 4     & 4     & 16 \\
		2     & 5     & 10    & 100 \\
		\rowcolor[rgb]{ .851,  .851,  .851} 3     & 6     & 18    & 324 \\
		4     & 7     & 28    & 784 \\
		\rowcolor[rgb]{ .851,  .851,  .851} 5     & 8     & 40    & 1600 \\
		6     & 9     & 54    & 2916 \\
		\rowcolor[rgb]{ .851,  .851,  .851} 7     & 10    & 70    & 4900 \\
		\bottomrule
	\end{tabular}%
	\label{tab:excel}%
\end{table}%

% Table generated by Excel2LaTeX from sheet 'Planilha1'
\begin{table}[htbp]
	\centering
	\caption{Add caption}
	\begin{tabular}{m{2cm}|m{2cm}|m{2cm}|m{2.3cm}}
		\toprule
		\multicolumn{4}{c}{\textbf{Dados referentes a nada, apenas para teste}} \\
		\midrule
		\midrule
		\multicolumn{1}{c|}{\textcolor[rgb]{ 1,  0,  0}{\textbf{A}}} & \multicolumn{1}{c|}{\textcolor[rgb]{ 1,  0,  0}{\textbf{B}}} & \multicolumn{1}{c|}{\textcolor[rgb]{ 1,  0,  0}{\textbf{C}}} & \multicolumn{1}{c}{\textcolor[rgb]{ 1,  0,  0}{\textbf{D}}} \\
		\midrule
		\rowcolor[rgb]{ .851,  .851,  .851} 1     & 4     & 4     & R\$ 16,00 \\
		\midrule
		2     & 5     & 10    & R\$ 100,00 \\
		\midrule
		\rowcolor[rgb]{ .851,  .851,  .851} 3     & 6     & 18    & R\$ 324,00 \\
		\midrule
		4     & 7     & 28    & R\$ 784,00 \\
		\midrule
		\rowcolor[rgb]{ .851,  .851,  .851} 5     & 8     & 40    & R\$ 1.600,00 \\
		\midrule
		6     & 9     & 54    & R\$ 2.916,00 \\
		\midrule
		\rowcolor[rgb]{ .851,  .851,  .851} 7     & 10    & 70    & R\$ 4.900,00 \\
		\bottomrule
	\end{tabular}%
	\label{tab:addlabel}%
\end{table}%


% Table generated by Excel2LaTeX from sheet 'Planilha1'
\begin{table}[h]
	\centering
	\caption{Add caption}
	\begin{tabular}{m{2cm}|m{2cm}|m{2cm}|m{2.3cm}}
		\toprule
		\multicolumn{4}{c}{\textbf{Dados referentes a nada, apenas para teste}} \\
		\midrule
		\midrule
		\textcolor[rgb]{ 1,  0,  0}{\textbf{A}} & \textcolor[rgb]{ 1,  0,  0}{\textbf{B}} & \textcolor[rgb]{ 1,  0,  0}{\textbf{C}} & \multicolumn{1}{c}{\textcolor[rgb]{ 1,  0,  0}{\textbf{D}}} \\
		\midrule
		1     & 4     & 4     & R\$ 16,00 \\
		\midrule
		2     & 5     & 10    & R\$ 100,00 \\
		\midrule
		3     & 6     & 18    & R\$ 324,00 \\
		\midrule
		4     & 7     & 28    & R\$ 784,00 \\
		\midrule
		5     & 8     & 40    & R\$ 1.600,00 \\
		\midrule
		6     & 9     & 54    & R\$ 2.916,00 \\
		\midrule
		7     & 10    & 70    & R\$ 4.900,00 \\
		\bottomrule
		\bottomrule
	\end{tabular}%
	\label{tab:addlabel}%
\end{table}%



O progresso tecnológico no mundo teve como marco inicial a descoberta da eletricidade. Diante de uma evolução tão importante, a humanidade pode experimentar um período de grandes descobertas. 
Os motores elétricos têm uma contribuição muito importante para o desenvolvimento da sociedade, seja economicamente e socialmente, fortalecendo o crescimento e evolução das indústrias.
Os primeiros motores foram desenvolvidos pela empresa alemã \textit{Siemens}, que funcionava através da corrente contínua. Tempos mais tarde após a descoberta da corrente alternada os motores de indução trifásicos foram construídos. Os motores de corrente alternada possuem algumas vantagens se comparados aos motores de corrente contínua como simplicidade, baixo custo e máxima eficácia com baixa manutenção. Quanto à classificação, os motores de indução podem ser síncronos, onde a velocidade de rotação é sincronizada com a frequência da rede de alimentação e os motores assíncronos, cuja a velocidade é próxima da velocidade síncrona \cite{Kosow1993}. 

A Equação \eqref{eq:01} descreve o sistema modelado... blá...blá blá 

\begin{equation}
	\label{eq:01}
	\dfrac{10}{s^{2}} 
\end{equation}

\begin{equation}
	%\nonumber
	{X}[{{z}}_k]=\sum^{N-1}_{{n=0}}{{x}[{n}]{(r_0e^{i{\theta }_0})}^{-n}{(R_0e^{i{\varphi }_0})}^{-nk}}, ~~ k=0,1,2,\dots ,M-1 \textrm{}
	%\tag{\ref{eq:19}}
\end{equation}

Nas aplicações industriais, os motores de indução trifásicos (MIT) são predominantes, e como resultados, são destinados ao acionamento de uma grande variedade de cargas. A observação de funcionamento dessas máquinas em condições reais de operação é fundamental para obter informações de demanda de corrente, potência, aquecimento, eficiência e entre outros.

Uma alternativa para obter informações reais de funcionamento de um motor de indução é a criação de um dispositivo que proporcione em laboratório simulações de cargas com diversos  perfis.  As simulações de cargas reais proporcionadas ao motor de indução podem ser impostas por meio de um dispositivo de frenagem com controle eletrônico, sendo capaz de proporcionar  variações de cargas  prédefinidas. Para promover o controle de força de frenagem, a utilização de um dispositivo de fricção como alternativa  implica na necessidade de materiais de elevada resistência ao desgaste, porém  a utilização de um freio por correntes induzidas dispensa essa necessidade, já que a força exercida para a frenagem é proporcionada pelo campo magnético gerado pelas bobinas. O freio eletromagnético por correntes induzidas tem seu princípio de funcionamento o campo magnético gerado pelas bobinas atuando sobre um disco fixado ao motor, desta forma ao variar a potência entregue às bobinas é possível obter uma resposta de frenagem variável \cite{Nolasco2013}. 
Para as simulações de cargas, a utilização de um sistema embarcado  permite  automatizar  diferentes perfis de cargas aplicados ao motor de indução, desta forma, a criação de situações de operações próximas do real funcionamento em campo  torna os ensaios mais significativos. 


Este trabalho irá discutir o desenvolvimento de uma bancada didática para simulação de carga em motores elétricos de indução. O sistema funciona com base em um freio eletromagnético para acoplamento ao motor e uma unidade de controle responsável por simular diversos tipos de cargas mecânicas. O conhecimento de como estes motores se comportam em campo é alvo de bastante estudos e de extrema importância para os trabalhos realizados na matéria de Máquinas Elétricas oferecidas em faculdades de Engenharia Elétrica. Realizar experimentos em motores de indução  nos seus locais de operação nem sempre é possível devido ao risco de acidentes e em alguns casos o ambiente é insalubre.  Os laboratórios para ensaios e simulações de cargas aplicados em motores equipado com o simulador de carga para o motor de indução, permitirá as pessoas interessadas no assunto, maiores conhecimentos didáticos nas realizações de estudos e experiências práticas.


\section{Considerações Iniciais}

Os motores de indução são considerados elementos indispensáveis na maioria dos setores industriais. Para observar o comportamento do motor durante o seu funcionamento, sabendo que nem sempre é possível ter acesso ao seu local de operação, uma alternativa é executar ensaios e simulações de cargas em laboratórios.  

É comum a utilização de um Gerador de corrente contínua (CC) funcionando como carga para motores de indução. Os ensaios realizados nos motores de indução em laboratórios são feitos principalmente através de acoplamento do motor de indução trifásico (MIT) a um Gerador CC. Este sistema apresenta alguns problemas durante o seu funcionamento em rotações baixas, pois o gerador não consegue manter uma corrente constante para suas resistências que estão funcionando como carga \cite{Denardi2013}. Outra observação importante a ser feita é que o preço dos Geradores CC são elevados. Para resolver problemas como estes, o projeto proposto é desenvolvido utilizando um freio eletromagnético que tem seu princípio de funcionamento através das correntes de \textit{Foucault}. O protótipo de uma bancada didática para montagem do projeto é descrita de forma detalhada através de imagens criadas em \textit{software} 3D, permitindo a quem interessar a realização de sua montagem. Todas as informações necessárias para replicar o simulador de carga para motor de indução estão disponíveis neste trabalho. A construção dos circuitos eletrônicos e montagem do freio eletromagnético estão detalhadas nos apêndices.

Para executar as simulações de cargas, foi desenvolvido uma Unidade de Controle onde o usuário poderá determinar diferentes tipos de cargas para os experimentos do motor de indução. Os parâmetros de configurações podem ser observados através de um \textit{display} e alterados com os botões conforme a necessidade do usuário.
 

\section{Organização da monografia}

\begin{description}
	\item[Capítulo \ref{cap:02} --] É descrito os objetivos deste trabalho, bem como a apresentação do problema e sua justificativa. É realizada, também, a contextualização e considerações iniciais.
	
	\item[Capítulo \ref{cap:03} --] É apresentada a fundamentação teórica com alguns conceitos sobre a conversão eletromagnética, funcionamento de motores de indução e freios por correntes induzidas. A apresentação sobre sistemas embutidos também é apresentado neste capítulo.
	
	\item[Capítulo \ref{cap:04} --] A descrição do projeto é feito neste capítulo, apresentando todo o processo de elaboração como: a construção da bancada; montagem dos circuitos eletrônicos de controle; arrumação e montagem do quadro de controle. Também é apresentado neste capítulo o \textit{software} desenvolvido para o funcionamento do simulador.
	
	\item[Capítulo \ref{cap:05} --] São apresentados e discutidos os principais resultados obtidos, bem como propostas para trabalhos futuros.
	
	\item[Capítulo \ref{cap:06} --]  São apresentadas as considerações finais sobre o trabalho.
	
	\item[Referências Bibliográficas --] São apresentadas as referências bibliográficas utilizadas no trabalho.
	
	\item[Apêndice \ref{apendice:01} --] Apêndice contendo o \textit{layout}, esquemático e relação de componentes dos principais circuitos desenvolvidos para o projeto.
	
	\item[Anexo \ref{anexo:01} --] Anexo contendo \textit{datasheets} dos principais circuitos integrados utilizados para o desenvolvimento do projeto.
	
\end{description}

%\newpage
%\thispagestyle{empty}