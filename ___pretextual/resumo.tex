% o texto deve começar imediatamente na próxima linha a fim de evitar identação
Os motores de indução tem um papel extremamente importante para o desenvolvimento técnico industrial. O entendimento sobre os aspectos práticos do funcionamento dessas máquinas elétricas são de grande importância para estudantes, profissionais e pesquisadores da área de Elétrica. Muitos dos aspectos práticos desses motores são observados apenas quando estão em operação, porém, nem sempre é possível a realização de experimentos em ambiente onde os motores estão naturalmente instalados. Nesse sentido surge a necessidade da criação de um ambiente laboratorial onde seja possível a realização de experimentos em condições de operações similares às obtidas no ambiente industrial. Diante dessas questões, esse trabalho tem por objetivo apresentar o desenvolvimento do protótipo de uma bancada didática que possibilite a execução de simulações de cargas em motores de indução. O protótipo baseia-se no conceito de freio magnético, cuja implementação permite que o motor seja submetido a diversos níveis de carga, possibilitando portanto, simular o comportamento de diversos sistemas reais. A abordagem metodológica consiste, essencialmente, nas seguintes etapas: construção da bancada de fixação do motor de indução; desenvolvimento do freio eletromagnético; implementação dos circuitos eletrônicos responsáveis pelo controle do sistema de frenagem; montagem do sistema para utilização em laboratório. Neste trabalho são apresentados todos os detalhes referentes à construção do protótipo, desde a estrutura necessária até os materiais e métodos utilizados. Para a validação do protótipo foram realizados alguns experimentos onde o motor foi submetido a diversos regimes de cargas. Os resultados são apresentados em forma de gráficos que mostram o perfil de variação do estímulo de frenagem como um sinal de entrada e a curva de variação da rotação do eixo do motor como um sinal de saída. A partir das observações a cerca desses resultados foi possível identificar o comportamento de não linearidade do sistema de frenagem. O protótipo pode, além de ser utilizado para atividades de ensino, pode também servir de base para o desenvolvimento de outros trabalhos, no que diz respeito às questões de melhoramento do sistema, como também a investigação de novas questões a cerca do funcionamento dos motores de indução.

\palavraschave{Motor de indução, Freio eletromagnético, bancada didática, sistemas embutidos}
% ATENÇÃO! Obrigatório um espaço entre a última linha do resumo e o comando \palavraschave{...}
