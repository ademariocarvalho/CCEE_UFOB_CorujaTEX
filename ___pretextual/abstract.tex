% o texto deve começar imediatamente na próxima linha a fim de evitar identação
Induction motors have an extremely important role in industrial technical development. The understanding of the practical aspects of the operation of these electric machines is of great importance for students, professionals and researchers in the Electrical area. Many of the practical aspects of these motors are only observed when they are in operation, however, it is not always possible to conduct experiments in an environment where the motors are naturally installed. In this sense, the need arises to create a laboratory environment, where it is possible to perform experiments in conditions of operations similar to those obtained in the industrial environment. In these questions, this work aims to present the development of the prototype of a didactic workbench that enables the execution of load simulations in induction motors. The prototype is based on the concept of magnetic brake, whose implementation allows the motor to be subjected to different load levels, making it possible to simulate the behavior of several real systems. The methodological approach consists essentially of the following steps: construction of the induction motor mounting bench; development of the electromagnetic brake; implementation of the electronic circuits responsible for controlling the braking system; making a systema for using in the laboratory. In this work all the details regarding the construction of the prototype are presented, from the necessary structure to the materials and methods used. For the validation of the prototype, some experiments were performed in which the motor was subjected to different load regimes. The results are presented in the form of graphs showing the braking stimulus variation profile as an input signal and the curve of the rotation of the motor axis as an output signal. From the observations about these results, it was possible to identify the nonlinearity behavior of the braking system. The prototype can, in addition to being used for teaching activities, can also serve as a basis for the development of other works, as regards system improvement issues, as well as the investigation of new issues regarding the induction.

\keywords{Induction motor, Electromagnetic brake, didactic bench, embedded systems}
% ATENÇÃO! Obrigatório um espaço entre a última linha do abstract e o comando \keywords{...}