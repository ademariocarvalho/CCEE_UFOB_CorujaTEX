\addcontentsline{toc}{chapter}{Apêndice}
\chapter{ALGORÍTIMO PARA ESTIMAR A VELOCIDADE DO MOTOR}
\label{apendice:01}

\section{Algorítmo da STCZT no Matlab}
\definecolor{verde}{RGB}{25, 90, 25}
\lstset{
		language=Matlab,
		extendedchars=true, % permite acentos
		basicstyle=\ttfamily\scriptsize, 
		keywordstyle=\color{blue}, 
		stringstyle=\color{purple}, 
		commentstyle=\color{verde}, 
		extendedchars=true, 
		showspaces=false, 
		showstringspaces=false, 
		numbers=left,
		numberstyle=\tiny,
		breaklines=true, 
		backgroundcolor=\color{white},
		breakautoindent=true, 
		captionpos=t, %top caption
		xleftmargin=10.5mm,
		tabsize=3,
		framexleftmargin=9.5mm,
		framexrightmargin=-1.0mm, 
		frame=shadowbox, %single
		rulesepcolor=\color{gray},
		%numberbychapter=false % deve ser usado antes do begin{document}..yseu na classe...insere numeração sequencial
}

%\lstset{framexleftmargin=6mm, 
%		frame=shadowbox, %single
%		rulesepcolor=\color{gray},
%		numberbychapter=true
%}


\begin{lstlisting}[caption= Algoritimo CZT no Matlab, label=prog:03]
	clc;
	NP=6*512;
	% fs=10000;
	fs=2*5000;
	Ts=1/fs;
	% dx=dfs10k;
	dados=data(:,3); %buffer_60Hz_125_p;  %_amp250_30Hz_variavel_240_025;
	%dados= dfs10k;  %corrente_fs_5k;
	dados=dados - mean(dados);
	NT=length(dados);
	T=NT*fs^-1;
	dp=50;
	d=dp*NP/100;  %4*128;
	NZ=floor((d-NP+NT)/d);
	s=1;
	% f0=25;
	% f1=f0-7;
	% f2=f0+7;
	m=NP;
	h=hanning(NP);
	fredez=zeros(1,NZ);
	dx=dados;
	t_p=(1:NZ)*(d/fs);
	tic;
	resfft=fs/NP;
	for i=1:NZ
	x=dx(s:s+NP-1).*h; 
	y = fft(x,NP); 
	[val idx]=max(abs(y));
	%f0=(idx)*resfft;
	
	%   f0=28;
	%   f1=f0-5;
	%   f2=f0+5;
	f1=22;
	f2=32;
	
	w=exp(-j*2*pi*(f2-f1)/(m*fs));                          % Set complex ratio
	a=exp(j*2*pi*f1/fs);                                    % Set complex starting point
	z=czt(x,m,w,a);                                         % Compute using CZT function
	abs_z=abs(z);
	[l c]=max(abs_z);
	fredez(i)=f1+(c-1)*((f2-f1)/NP);
	s=s+d;
	end
	
	tempo=toc;
	% hold on;
	% plot(t,fredez,'r');
	% figure;
	% plot(t,fsh,'r');
	% xlabel('Tempo(s)');
	% ylabel('Frequencia');
	% title(['STCZT::: Janela:<',num2str(NP),' pontos> Desclocamento:<',num2str(d),' pontos = ',num2str(100*d/NP),'%>']);
	tmax=60;
	figure;
	subplot(2,1,2);
	plot(t_p,60*fredez,'r');
	axis([0 tmax 1740 1790])
	xlabel('Tempo(s)');
	ylabel('Rotacao (RPM)');
	
	subplot(2,1,1);
	pwm_i=data(:,2);
	pwm_i=pwm_i-mean(pwm_i(1:100));
	cmax=60;
	cmin=0;
	pmax=mean(pwm_i)*2,
	pwm_f=cmin+pwm_i*(cmax/pmax);
	tf=0:Ts:length(pwm_f)*Ts - Ts;
	plot(tf,pwm_f,'b');
	axis([0 tmax -3 45])
	% hold on
	% pulso_f=data(:,1); 
	% pulso_f=pulso_f-mean(pulso_f(1:10));
	% plot(tf, pulso_f,'g');
	xlabel('Tempo(s)');
	ylabel('Carga de frenagem (%)');
	% title(['STCZT::: Janela:<',num2str(NP),' pontos> Desclocamento:<',num2str(d),' pontos = ',num2str(100*d/NP),'%>']);
	% clearvars -except -regexp _p$ % limpa todas vari?veis exceto aquelas que terminem com "_p" -> Adem?rio
	%  
	% grid;
	hold off
	
\end{lstlisting}

\section{Seção do apêndice}
\lipsum %% gera texto apenas para teste